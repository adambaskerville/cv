%----------------------------------------------------------------------------------------
%	SECTION TITLE
%----------------------------------------------------------------------------------------

\cvsection{Experience}

%----------------------------------------------------------------------------------------
%	SECTION CONTENT
%----------------------------------------------------------------------------------------

%\begin{cvskills}

%------------------------------------------------
\cvparagraph{
	
\cventry
{Department of Chemistry, School of Life Sciences, University of Sussex} % Institution
{\textbf{Postdoctoral Research Fellow}} % Degree
{Brighton, UK} % Location
{May. 2018 - Current} % Date(s)
{\textbf{Projects:}}

\begin{itemize}
	\item \textbf{3Body Library}:
	\begin{itemize}
		\item Developed extensive library of high accuracy programs applicable to solving the Schr\"{o}dinger equation for any three-particle system, exploring new quantum chemical physics using their wavefunctions. All programs thoroughly documented for use by students.
		\item Utilizes C\texttt{++}, Python, CUDA and Maple with Bash orchestrating to allow operations between each language. Non-linear parameter, NLP, optimisation was extensively used and all programs designed to use double, quadruple, octuple and ball arithmetic precision.
	\end{itemize}
	\item \textbf{Density-Energy Space}:
	\begin{itemize}
		\item Developed new physical framework which has shown a direct relationship between radial electron density and the total energy for quantum mechanical systems; a realisation of the Hohenberg and Kohn theorem. \textit{Submitted for publication in Nature}.
		\item Utilised communication between Maple and C\texttt{++} which conducted numerical integration in parallel over a unit hypercube using quadruple precision.
	\end{itemize}
	\item \textbf{Electron Correlation in Electronic Structure Theory}:
	\begin{itemize}
		\item Calculated the current best electron correlation data for two-electron atoms through the Fully Correlated (FC) and Hartree Fock (HF) programs I developed in the 3Body library.
		\item Developed highly optimized, vectorized and parallelized C program to rapidly calculate millions of two-electron integrals using 200-digit precision with rigorous error bounds for each integral.
		\item Collaborators at Bristol University using this data to develop functional for use in Density Functional Theory (DFT).
	\end{itemize}
	\item \textbf{Density Functional Theory Investigation}:
	\begin{itemize}
		\item Implemented work of Colle and Salvetti (CS), Lee, Yang and Parr (LYP) and other DFT functionals to interface with my high accuracy FC and HF wavefunctions.
		\item Showed that the CS and popular LYP functional work because of inaccurate function fit conducted by CS; questioning the physical basis behind the most popular functional used in DFT. \textit{Submitted for publication in Royal Society Open Science}.
		\item Currently developing functional for weakly bound systems which DFT struggles to calculate. \textit{In preparation for publication in Physical Chemistry Chemical Physics}.
	\end{itemize}
	\item \textbf{Weakly Bound and Excited States}:
	\begin{itemize}
		\item Implemented theoretical methodologies to analyse asymmetric behaviour of fermions in two-electron systems. This has been used to show quasi-bound behaviour in atoms near electron detachment, and to calculate highly excited Rydberg states. \textit{In preparation for publication}.
		\item Developed a deep-learning inspired GPU technique using PyTorch to successfuly circumvent computational difficulties owing to ill-conditioned matrices.
	\end{itemize}
\end{itemize}

{\textbf{Roles and Responsibilities:}}
\begin{itemize}
		\item Lead programmer in the research group requiring knowledge across wide variety of scientific problems.
		\item Lab manager responsible for maintaining computational resources in the research group.
		\item Teacher and mentor to numerous MChem and PhD students. 
		\item Write papers for publication and present research in the form of oral presentations and posters at national and international conferences.
		\item Helped develop `Quantum Leap', an interactive game to explain quantization at outreach events.
		\item \textbf{Lecturer}:
		\begin{itemize}
			\item Created and taught my own scientific Python programming course for staff and students.
			\item Lectured maths and data analysis for undergraduate students in the chemistry department.
		\end{itemize}
		\item \textbf{Associate Tutor}:
		\begin{itemize}
			\item Assisted in teaching mathematics, data analysis, point group symmetry, bonding, spectroscopy, and computational chemistry across multiple year groups from foundation to final year students in bioscience and chemistry departments. This involved problem-based workshops and marking.
		\end{itemize}
\end{itemize}

%My current goal is to build upon my scientific training by broadening my understanding of applied machine and deep learning to problems in the natural sciences. A position at DeepMind offers the perfect opportunity to utilize and improve my skills as a scientist and developer.
}

%------------------------------------------------

%\end{cvskills}