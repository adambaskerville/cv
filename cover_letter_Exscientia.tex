%%%%%%%%%%%%%%%%%%%%%%%%%%%%%%%%%%%%%%%%%
% Important note:
% This template must be compiled with XeLaTeX, the below lines will ensure this
%!TEX TS-program = xelatex
%!TEX encoding = UTF-8 Unicode
%
%%%%%%%%%%%%%%%%%%%%%%%%%%%%%%%%%%%%%%%%%

%----------------------------------------------------------------------------------------
%	PACKAGES AND OTHER DOCUMENT CONFIGURATIONS
%----------------------------------------------------------------------------------------

\documentclass[a4paper]{awesome-cv} % A4 paper size by default, use 'letterpaper' for US letter

\geometry{left=2cm, top=1cm, right=2cm, bottom=1.7cm, footskip=.5cm} % Configure page margins with geometry
 
\fontdir[fonts/] % Specify the location of the included fonts

% Color for highlights
\colorlet{awesome}{awesome-red} % Default colors include: awesome-emerald, awesome-skyblue, awesome-red, awesome-pink, awesome-orange, awesome-nephritis, awesome-concrete, awesome-darknight
%\definecolor{awesome}{HTML}{CA63A8} % Uncomment if you would like to specify your own color

% Colors for text - uncomment and modify
%\definecolor{darktext}{HTML}{414141}
%\definecolor{text}{HTML}{414141}
%\definecolor{graytext}{HTML}{414141}
%\definecolor{lighttext}{HTML}{414141}

\renewcommand{\acvHeaderSocialSep}{\quad\textbar\quad} % If you would like to change the social information separator from a pipe (|) to something else

%----------------------------------------------------------------------------------------
%	PERSONAL INFORMATION
%	Comment any of the lines below if they are not required
%----------------------------------------------------------------------------------------

\name{Dr. Adam Luke}{Baskerville}
\mobile{(+44) 07896 941 917}

\email{ab695@sussex.ac.uk}
\homepage{adambaskerville.github.io/}
%\twitter{@AdamBask}

%----------------------------------------------------------------------------------------
%	RECIPIENT/POSITION/LETTER INFORMATION
%	All of the below lines must be filled out
%----------------------------------------------------------------------------------------

\letterdate{\today} % The date on the letter, default is the date of compilation

\lettertitle{} % The title of the letter

\letteropening{} % How the letter is opened

\letterclosing{Sincerely,} % How the letter is closed

%\letterenclosure[Attached]{Curriculum Vitae} % Any enclosures with the letter

\makecvfooter{\today}{Adam Luke Baskerville~~~·~~~Cover Letter}{} % Specify the letter footer with 3 arguments: (<left>, <center>, <right>), leave any of these blank if they are not needed
  
%----------------------------------------------------------------------------------------

\recipient{~}{\vspace{-2.5cm}~} 

\begin{document}

\makecvheader % Print the header

\makelettertitle % Print the title

%----------------------------------------------------------------------------------------
%	LETTER CONTENT
%----------------------------------------------------------------------------------------
\vspace{-0.3cm}
\begin{cvletter}

I am a research fellow (EPSRC funded, EP/R011265/1) at the University of Sussex in collaboration with Bristol University, investigating the fundamentals of electron correlation with applications to density functional theory (DFT). I actively use computers to solve complicated scientific problems and know I can contribute to research and innovation within industry. My research and expertise are an excellent fit for this post, drawing inspiration from physics, chemistry and state of the art computational techniques. I have been able to build a versatile toolbox of knowledge applicable to a variety of scientific and computational problems such as utilizing GPU programming techniques on a Titan V GPU awarded from NVIDIA.


Through my research I developed a library of high accuracy C\texttt{++} and Python computer codes to solve the non-relativistic, time-independent Schr\"{o}dinger equation for three-particle atomic and molecular systems, using a novel series solution method. I employed high accuracy computational techniques including DoubleDouble and QuadDouble data types, ball arithmetic and CPU/GPU parallel programming. These programs explored beyond the standard Born Oppenheimer and orbital approximations, to discover for the first time that nuclear motion is correlated in heteronuclear diatomic ions (`\textit{Physical Review A}'). I also confirmed that secondary Coulomb holes exist for various two-electron atomic systems (`\textit{Royal Society Open Science}'). I investigated the bound state stability of two-electron atoms where the critical nuclear charge for binding two electrons to a nucleus was calculated for the first time using Hartree Fock theory (`\textit{Philosophical Transactions of the Royal Society A}' and `\textit{Advances in Quantum Chemistry.}')


During my postdoctoral position I have calculated the most accurate electron correlation data to date for two-electron systems, using the high accuracy Fully Correlated (FC) and Hartree Fock (HF) computer codes that I developed. I am using these data to develop a new correlation functional for use in DFT with a focus on weakly bound systems, which DFT struggles to capture. Using a combination of FC and HF data, a physically motivated functional form is being fitted to the high accuracy data for weakly bound systems. These new, exciting results are being prepared for publication in `\textit{Physical Review Letters}' and `\textit{Physical Chemistry Chemical Physics}'. Recently I developed a physical framework I term `density-energy space' which has revealed for the first time, direct relationships between radial electron density and the total energy for a quantum mechanical system; a realisation of the Hohenberg and Kohn theorem. These results are being prepared for publication in `\textit{Nature}'. I am currently building on a deep learning inspired GPU technique I developed to calculate generalized eigenvalues for a very ill-conditioned matrix eigenvalue problem. This has proven pivotal in revealing unseen quasi-bound behaviour for few-electron systems near electron detachment.


I have enjoyed presenting my research at a variety of national and international conferences including the International Society of Theoretical Chemical Physics (ISTCP) and the International Meeting on Atomic and Molecular Physics and Chemistry (IMAMPC). Last year I was selected to present my research at STEM for Britain at the Houses of Parliament where I discussed `\textit{Electron Correlation in the Real World}' with members of parliament.


I have extensive experience working in an academic environment with six years of teaching and lecturing experience, covering a variety of workshops from mathematics and data analysis, quantum mechanics, bonding, spectroscopy and computational chemistry. I independently created and taught my own seven session scientific programming course inspired by feedback from staff and students in the chemistry department who wanted to learn programming. The sessions included data types, control flow, NumPy, input/output, physical chemistry and practical machine learning for chemists.


Throughout my academic career as a teacher and mentor to numerous MChem and PhD students, it has been an honour to help and see their confidence and knowledge grow. Research often comes with a steep learning curve and I provide introductory guides and tutorials on the research codes for students. 


My scientific training and computational background mean that I would be an effective scientific developer, researcher and programmer. I have been looking for a position which allows me to pursue my passion for solving scientific problems using computers.

\vspace{-0.2cm}
\end{cvletter}

%----------------------------------------------------------------------------------------

\makeletterclosing % Print the signature and enclosures

\end{document}